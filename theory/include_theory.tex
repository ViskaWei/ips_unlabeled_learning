%% Include this file in LED_ips_nn.tex to add the theory section
%% Add before \bibliography:
%%
%% %% Include this file in LED_ips_nn.tex to add the theory section
%% Add before \bibliography:
%%
%% %% Include this file in LED_ips_nn.tex to add the theory section
%% Add before \bibliography:
%%
%% %% Include this file in LED_ips_nn.tex to add the theory section
%% Add before \bibliography:
%%
%% \input{theory/include_theory}
%%
%% Or copy the contents directly into your main file.

%% ============================================
%% THEORETICAL ANALYSIS
%% ============================================

\section{Theoretical Analysis}
\label{sec:theory}

We establish the theoretical foundations for learning from unlabeled ensemble data. Our analysis addresses identifiability, consistency, and convergence rates.

\subsection{Identifiability}

A fundamental question is whether the potentials $(\Phi^*, V^*)$ can be uniquely determined from the data distribution.

\begin{definition}[Identifiability]
The pair $(\Phi^*, V^*)$ is \emph{identifiable} if for any $(\Phi, V)$ with $\mathcal{E}_\infty(\Phi, V) = \mathcal{E}_\infty(\Phi^*, V^*)$, we have $\Phi = \Phi^* + c_1$ and $V = V^* + c_2$ for constants $c_1, c_2$.
\end{definition}

\begin{definition}[Coercivity]
The coercivity condition holds with constant $c_H > 0$ if:
$$
\E{\frac{1}{N}\sum_i \left|\nabla \delta V(X_t^i) + \frac{1}{N}\sum_j \nabla \delta\Phi(X_t^i - X_t^j)\right|^2} \geq c_H \left(\|\nabla \delta V\|_{L^2}^2 + \|\nabla \delta\Phi\|_{L^2}^2\right)
$$
for all zero-mean perturbations $\delta V, \delta\Phi$.
\end{definition}

\begin{theorem}[Identifiability]\label{thm:main_ident}
Under assumptions (A1)-(A3) and the coercivity condition, $(\Phi^*, V^*)$ is identifiable.
\end{theorem}

The proof shows that the loss residual $\mathcal{E}(\Phi,V) - \mathcal{E}(\Phi^*,V^*)$ is lower bounded by the coercivity constant times the squared distance, so zero residual implies zero distance.

\begin{proposition}[Gaussian Coercivity]
For i.i.d. Gaussian initial conditions $X_0^i \sim \mathcal{N}(0, I_d)$, the coercivity constant satisfies $c_H \geq 0.48$ for $d=1$, $c_H \geq 0.87$ for $d=2$, and $c_H \geq 0.73$ for $d=3$.
\end{proposition}

\subsection{Consistency and Convergence Rates}

\begin{theorem}[Consistency]\label{thm:consist}
Let $(\hat{\Phi}_n, \hat{V}_n)$ minimize $\mathcal{E}_{\mathcal{D}}$ with $n = ML$ samples. Under coercivity:
$$
\|\nabla\hat{\Phi}_n - \nabla\Phi^*\|_{L^2} + \|\nabla\hat{V}_n - \nabla V^*\|_{L^2} \xrightarrow{P} 0.
$$
\end{theorem}

\begin{theorem}[Rate]\label{thm:rate}
For neural network estimators with width $W$ and depth $D$, the total error decomposes as:
$$
\text{Error} \leq \underbrace{O(W^{-2(s-1)/d})}_{\text{approximation}} + \underbrace{O\left(\frac{WD}{n}\right)}_{\text{estimation}} + \underbrace{O(\Delta t)}_{\text{discretization}}.
$$
Optimizing $W \asymp n^{d/(2s+d-2)}$ gives rate $n^{-2(s-1)/(2s+d-2)}$ for $C^s$ potentials.
\end{theorem}

\begin{theorem}[Minimax Lower Bound]\label{thm:lower}
For $C^s$ potential classes:
$$
\inf_{\hat{\Phi}} \sup_{\Phi^*} \E{\|\nabla\hat{\Phi} - \nabla\Phi^*\|^2} \geq c \cdot n^{-2(s-1)/(2s+d)}.
$$
\end{theorem}

The upper and lower bounds match for $s=2$ (twice differentiable potentials), confirming minimax optimality.

%% Proofs are in Appendix
\textit{Proofs are provided in Appendix \ref{app:proofs}.}

%%
%% Or copy the contents directly into your main file.

%% ============================================
%% THEORETICAL ANALYSIS
%% ============================================

\section{Theoretical Analysis}
\label{sec:theory}

We establish the theoretical foundations for learning from unlabeled ensemble data. Our analysis addresses identifiability, consistency, and convergence rates.

\subsection{Identifiability}

A fundamental question is whether the potentials $(\Phi^*, V^*)$ can be uniquely determined from the data distribution.

\begin{definition}[Identifiability]
The pair $(\Phi^*, V^*)$ is \emph{identifiable} if for any $(\Phi, V)$ with $\mathcal{E}_\infty(\Phi, V) = \mathcal{E}_\infty(\Phi^*, V^*)$, we have $\Phi = \Phi^* + c_1$ and $V = V^* + c_2$ for constants $c_1, c_2$.
\end{definition}

\begin{definition}[Coercivity]
The coercivity condition holds with constant $c_H > 0$ if:
$$
\E{\frac{1}{N}\sum_i \left|\nabla \delta V(X_t^i) + \frac{1}{N}\sum_j \nabla \delta\Phi(X_t^i - X_t^j)\right|^2} \geq c_H \left(\|\nabla \delta V\|_{L^2}^2 + \|\nabla \delta\Phi\|_{L^2}^2\right)
$$
for all zero-mean perturbations $\delta V, \delta\Phi$.
\end{definition}

\begin{theorem}[Identifiability]\label{thm:main_ident}
Under assumptions (A1)-(A3) and the coercivity condition, $(\Phi^*, V^*)$ is identifiable.
\end{theorem}

The proof shows that the loss residual $\mathcal{E}(\Phi,V) - \mathcal{E}(\Phi^*,V^*)$ is lower bounded by the coercivity constant times the squared distance, so zero residual implies zero distance.

\begin{proposition}[Gaussian Coercivity]
For i.i.d. Gaussian initial conditions $X_0^i \sim \mathcal{N}(0, I_d)$, the coercivity constant satisfies $c_H \geq 0.48$ for $d=1$, $c_H \geq 0.87$ for $d=2$, and $c_H \geq 0.73$ for $d=3$.
\end{proposition}

\subsection{Consistency and Convergence Rates}

\begin{theorem}[Consistency]\label{thm:consist}
Let $(\hat{\Phi}_n, \hat{V}_n)$ minimize $\mathcal{E}_{\mathcal{D}}$ with $n = ML$ samples. Under coercivity:
$$
\|\nabla\hat{\Phi}_n - \nabla\Phi^*\|_{L^2} + \|\nabla\hat{V}_n - \nabla V^*\|_{L^2} \xrightarrow{P} 0.
$$
\end{theorem}

\begin{theorem}[Rate]\label{thm:rate}
For neural network estimators with width $W$ and depth $D$, the total error decomposes as:
$$
\text{Error} \leq \underbrace{O(W^{-2(s-1)/d})}_{\text{approximation}} + \underbrace{O\left(\frac{WD}{n}\right)}_{\text{estimation}} + \underbrace{O(\Delta t)}_{\text{discretization}}.
$$
Optimizing $W \asymp n^{d/(2s+d-2)}$ gives rate $n^{-2(s-1)/(2s+d-2)}$ for $C^s$ potentials.
\end{theorem}

\begin{theorem}[Minimax Lower Bound]\label{thm:lower}
For $C^s$ potential classes:
$$
\inf_{\hat{\Phi}} \sup_{\Phi^*} \E{\|\nabla\hat{\Phi} - \nabla\Phi^*\|^2} \geq c \cdot n^{-2(s-1)/(2s+d)}.
$$
\end{theorem}

The upper and lower bounds match for $s=2$ (twice differentiable potentials), confirming minimax optimality.

%% Proofs are in Appendix
\textit{Proofs are provided in Appendix \ref{app:proofs}.}

%%
%% Or copy the contents directly into your main file.

%% ============================================
%% THEORETICAL ANALYSIS
%% ============================================

\section{Theoretical Analysis}
\label{sec:theory}

We establish the theoretical foundations for learning from unlabeled ensemble data. Our analysis addresses identifiability, consistency, and convergence rates.

\subsection{Identifiability}

A fundamental question is whether the potentials $(\Phi^*, V^*)$ can be uniquely determined from the data distribution.

\begin{definition}[Identifiability]
The pair $(\Phi^*, V^*)$ is \emph{identifiable} if for any $(\Phi, V)$ with $\mathcal{E}_\infty(\Phi, V) = \mathcal{E}_\infty(\Phi^*, V^*)$, we have $\Phi = \Phi^* + c_1$ and $V = V^* + c_2$ for constants $c_1, c_2$.
\end{definition}

\begin{definition}[Coercivity]
The coercivity condition holds with constant $c_H > 0$ if:
$$
\E{\frac{1}{N}\sum_i \left|\nabla \delta V(X_t^i) + \frac{1}{N}\sum_j \nabla \delta\Phi(X_t^i - X_t^j)\right|^2} \geq c_H \left(\|\nabla \delta V\|_{L^2}^2 + \|\nabla \delta\Phi\|_{L^2}^2\right)
$$
for all zero-mean perturbations $\delta V, \delta\Phi$.
\end{definition}

\begin{theorem}[Identifiability]\label{thm:main_ident}
Under assumptions (A1)-(A3) and the coercivity condition, $(\Phi^*, V^*)$ is identifiable.
\end{theorem}

The proof shows that the loss residual $\mathcal{E}(\Phi,V) - \mathcal{E}(\Phi^*,V^*)$ is lower bounded by the coercivity constant times the squared distance, so zero residual implies zero distance.

\begin{proposition}[Gaussian Coercivity]
For i.i.d. Gaussian initial conditions $X_0^i \sim \mathcal{N}(0, I_d)$, the coercivity constant satisfies $c_H \geq 0.48$ for $d=1$, $c_H \geq 0.87$ for $d=2$, and $c_H \geq 0.73$ for $d=3$.
\end{proposition}

\subsection{Consistency and Convergence Rates}

\begin{theorem}[Consistency]\label{thm:consist}
Let $(\hat{\Phi}_n, \hat{V}_n)$ minimize $\mathcal{E}_{\mathcal{D}}$ with $n = ML$ samples. Under coercivity:
$$
\|\nabla\hat{\Phi}_n - \nabla\Phi^*\|_{L^2} + \|\nabla\hat{V}_n - \nabla V^*\|_{L^2} \xrightarrow{P} 0.
$$
\end{theorem}

\begin{theorem}[Rate]\label{thm:rate}
For neural network estimators with width $W$ and depth $D$, the total error decomposes as:
$$
\text{Error} \leq \underbrace{O(W^{-2(s-1)/d})}_{\text{approximation}} + \underbrace{O\left(\frac{WD}{n}\right)}_{\text{estimation}} + \underbrace{O(\Delta t)}_{\text{discretization}}.
$$
Optimizing $W \asymp n^{d/(2s+d-2)}$ gives rate $n^{-2(s-1)/(2s+d-2)}$ for $C^s$ potentials.
\end{theorem}

\begin{theorem}[Minimax Lower Bound]\label{thm:lower}
For $C^s$ potential classes:
$$
\inf_{\hat{\Phi}} \sup_{\Phi^*} \E{\|\nabla\hat{\Phi} - \nabla\Phi^*\|^2} \geq c \cdot n^{-2(s-1)/(2s+d)}.
$$
\end{theorem}

The upper and lower bounds match for $s=2$ (twice differentiable potentials), confirming minimax optimality.

%% Proofs are in Appendix
\textit{Proofs are provided in Appendix \ref{app:proofs}.}

%%
%% Or copy the contents directly into your main file.

%% ============================================
%% THEORETICAL ANALYSIS
%% ============================================

\section{Theoretical Analysis}
\label{sec:theory}

We establish the theoretical foundations for learning from unlabeled ensemble data. Our analysis addresses identifiability, consistency, and convergence rates.

\subsection{Identifiability}

A fundamental question is whether the potentials $(\Phi^*, V^*)$ can be uniquely determined from the data distribution.

\begin{definition}[Identifiability]
The pair $(\Phi^*, V^*)$ is \emph{identifiable} if for any $(\Phi, V)$ with $\mathcal{E}_\infty(\Phi, V) = \mathcal{E}_\infty(\Phi^*, V^*)$, we have $\Phi = \Phi^* + c_1$ and $V = V^* + c_2$ for constants $c_1, c_2$.
\end{definition}

\begin{definition}[Coercivity]
The coercivity condition holds with constant $c_H > 0$ if:
$$
\E{\frac{1}{N}\sum_i \left|\nabla \delta V(X_t^i) + \frac{1}{N}\sum_j \nabla \delta\Phi(X_t^i - X_t^j)\right|^2} \geq c_H \left(\|\nabla \delta V\|_{L^2}^2 + \|\nabla \delta\Phi\|_{L^2}^2\right)
$$
for all zero-mean perturbations $\delta V, \delta\Phi$.
\end{definition}

\begin{theorem}[Identifiability]\label{thm:main_ident}
Under assumptions (A1)-(A3) and the coercivity condition, $(\Phi^*, V^*)$ is identifiable.
\end{theorem}

The proof shows that the loss residual $\mathcal{E}(\Phi,V) - \mathcal{E}(\Phi^*,V^*)$ is lower bounded by the coercivity constant times the squared distance, so zero residual implies zero distance.

\begin{proposition}[Gaussian Coercivity]
For i.i.d. Gaussian initial conditions $X_0^i \sim \mathcal{N}(0, I_d)$, the coercivity constant satisfies $c_H \geq 0.48$ for $d=1$, $c_H \geq 0.87$ for $d=2$, and $c_H \geq 0.73$ for $d=3$.
\end{proposition}

\subsection{Consistency and Convergence Rates}

\begin{theorem}[Consistency]\label{thm:consist}
Let $(\hat{\Phi}_n, \hat{V}_n)$ minimize $\mathcal{E}_{\mathcal{D}}$ with $n = ML$ samples. Under coercivity:
$$
\|\nabla\hat{\Phi}_n - \nabla\Phi^*\|_{L^2} + \|\nabla\hat{V}_n - \nabla V^*\|_{L^2} \xrightarrow{P} 0.
$$
\end{theorem}

\begin{theorem}[Rate]\label{thm:rate}
For neural network estimators with width $W$ and depth $D$, the total error decomposes as:
$$
\text{Error} \leq \underbrace{O(W^{-2(s-1)/d})}_{\text{approximation}} + \underbrace{O\left(\frac{WD}{n}\right)}_{\text{estimation}} + \underbrace{O(\Delta t)}_{\text{discretization}}.
$$
Optimizing $W \asymp n^{d/(2s+d-2)}$ gives rate $n^{-2(s-1)/(2s+d-2)}$ for $C^s$ potentials.
\end{theorem}

\begin{theorem}[Minimax Lower Bound]\label{thm:lower}
For $C^s$ potential classes:
$$
\inf_{\hat{\Phi}} \sup_{\Phi^*} \E{\|\nabla\hat{\Phi} - \nabla\Phi^*\|^2} \geq c \cdot n^{-2(s-1)/(2s+d)}.
$$
\end{theorem}

The upper and lower bounds match for $s=2$ (twice differentiable potentials), confirming minimax optimality.

%% Proofs are in Appendix
\textit{Proofs are provided in Appendix \ref{app:proofs}.}
